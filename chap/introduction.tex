% Copyright (c) 2014,2016 Casper Ti. Vector
% Public domain.

\specialchap{序言}
%\pkuthssffaq % 中文测试文字。

随着物理学的发展,人们对于这个世界的理解也越来越深入。在粒子物理的相关研究中,组成世界的基本粒子一共有12种,分为6种夸克,3种带电轻子以及三种中微子。随着对于这十二种粒子的深入研究和理解,人们在量子场论的框架下构建了基于量子色动力学和弱点相互统一理论的标准模型(Standard Model, SM)。标准模型成功的解释了宇宙中物质的构成以及他们之间的相互作用,并被大量的实验所证明。尤其是2012年标准模型所预测的最后一种粒子——希格斯玻色子被发现,更是完美的符合了标准模型的预测。

在标准模型所描述的基本粒子中,中微子是其中性质十分的特殊,也是最神秘的一个。对中微子的研究最早可以追溯到1930年,奥地利物理学家泡利(Wolfgang Ernst Pauli)在一封解释$\beta$衰变问题的信中,提出了一种微小的电中性粒子以解释衰变中能谱连续的问题。这个假设被费米(Enrico Fermi)引入到了他的$\beta$衰变理论中\supercite{wilson1968fermi}。而后在1956年,Clyde Cowan和Frederick Reines首次确认了电子中微子的存在\supercite{cowan1991detection},于是中微子揭开了它隐藏的面纱,被人们所渐渐了解。$\mu$中微子于1962年被Brookhaven国家实验室所发现\supercite{danby1962observation},而后知道2000年,最后一种中微子$\tau$中微子存在的直接证据才被费米实验室所找到\supercite{kodama2001observation}。

标准模型中中微子被认为是没有质量的,然而过往和正在进行的大量中微子实验,如超级神冈(Super-Kamiokande)\supercite{fukuda1998evidence},萨德伯里中微子观测站(SNO)\supercite{ahmad2002direct}等,都观测到了中微子震荡现象。对这些现象最为自然的解释就是中微子是有质量的,这也可能标志着标准模型之外的,未被探索的新物理。

中微子既然可能有质量,那么它质量的来源便是一个十分有趣的问题。如果假设中微子是狄拉克费米子(Dirac fermions),那么为了使中微子产生质量,中微子与希格斯场相互作用的耦合系数会比夸克等粒子小12个量级,这种模型解释起来会变得十分的困难。另一种假设是中微子是马约拉纳费米子(Majorana fermions),即中微子自身是自己的反粒子,这种模型对比而言更为自然自洽。然而到现在为止,还没有足够的实验观测能够判断哪种模型更为正确。


% vim:ts=4:sw=4
