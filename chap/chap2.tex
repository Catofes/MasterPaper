% Copyright (c) 2014,2016 Casper Ti. Vector
% Public domain.

\chapter{探测器蒙特卡洛模拟}
\label{chapter:background}

为了寻找NLDBD这个极其稀有的事件,一个极低本底的探测环境是PandaXIII实验需要着重解决的问题。所以为了能够在探测器实际建造之前细致的研究探测器结构和材料对本底水平的研究,我们使用Geant4\supercite{Agostinelli:2002hh}作为主要的蒙特卡洛模拟框架,构建了模拟器的模型。通过这个模型我们可以细致的研究探测器各个组件以及环境对实验本底的贡献,并根据这些数据来调节探测器的各种各样参数以达成实验的设计目标。具体的模拟细节如下:

\section{目标元素及模拟工具}

因为探测器的铜壁相对较厚,而且铜壁的材料可以制作的较为纯净,在考虑到铜壁的屏蔽效应后,来自铜壁外的$\alpha$射线和$\beta$射线基本不可能到达探测器内部的灵敏区域,所以需要主要考虑到的本底辐射为探测器铜壁内部原件的各种辐射以及环境和材料中的$\gamma$射线。因为$^{136}Xe$的NLDBD事件释放出来的总能量$Q_{\beta\beta}=2458$keV,探测器设计的相对能量分辨率是3\%FWHW(半高全宽),因而我们定义能量窗口($Q_{\beta\beta}-2\sigma$, $Q_{\beta\beta}+2\sigma$)为能量敏感区域(Region Of Interest, ROI),其中$\sigma$为探测器在$Q_{\beta\beta}$处的绝对能量分辨率。计算可得ROI范围为2395keV到2520keV,所以衰变能量落在这个能量范围附近的元素便是我们感兴趣的元素。

根据各种放射性元素的衰变能量和在自然界中的丰度,再结合既往的$^{136}$XeNLDBD实验研究结果可以得到,我们关心的主要本底来自于$^{214}$Bi的gamma衰变,其能量为2447.8keV,以及来自于$^{208}$Tl的gamma衰变,其能量为2615.keV。这两种元素分别属于$^{238}$U和$^{232}Th$的衰变链中,而这两种放射性元素在自然界中大量的存在,绝大多数的材料都或多或少的含有它们。除了这两种元素,$^{60}$Co可能会释放出1.33MeV和1.17MeV的$\gamma$射线,虽然两者和能量约为2.5MeV,但是这两个射线相对独立,同时落在探测器敏感区域内的概率不大,在加上读出窗口的限制$^{60}$Co对实验的影响会变得更小,因而在后续的模拟中它并没有被着重研究。

\section{探测器结构及本底来源}

\section{探测器响应及读出触发}

% vim:ts=4:sw=4
