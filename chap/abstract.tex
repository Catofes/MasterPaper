% Copyright (c) 2014,2016 Casper Ti. Vector
% Public domain.

\begin{cabstract}
	\pkuthssffaq % 中文测试文字
	PandaXIII实验试图在中国锦屏地下实验室建造一个高压气氙时间漂移室探测器来寻找\xeots元素的无中微子双Beta衰变的事件。为了寻找这种半衰期大于$1\times10^{26}$年的超稀有事件,PandaXIII探测器需要在保证较高探测效率的同时,尽量减少本底水平。因而需要通过合理的探测器设计和材料选择以及高效的背景信号鉴别来提成探测器的探测效率。

	本文通过蒙特卡洛模拟研究了在PandaXIII实验设计的探测器中,不同组件中天然含有的\utte以及\thttt元素所衰变产生的本底贡献。其中探测器读出板 Micorbulk Micromegas 以及铆合探测器铜罐的不锈钢螺钉贡献了绝大部分的本底,探测器的其他组件以及周围环境则可忽略不计。总计探测器的本底水平为 $308\times 10^{-5}$ \ckky。

	相对于本底$\gamma$事件,NLDBD自身的径迹特征十分的鲜明,因此可以通过径迹的重建来压低背景信号。本文研究了使用深度卷积神经网络对模拟产生的NLDBD信号以及$\gamma$背景进行鉴别,该方法能够在保持47.5\%信号效率的同时压低背景175倍,相较于PandaXIII中期设计报告中的目标提高了62\%,取得了十分优异的结果。

	在PandaXIII模拟重建工作之外,本文也探究了在中微子能谱分析探究过程中,考虑到探测器分辨率本身随能量连续变化的情况时,使用GPU来加速Roofit的拟合过程的方法。得益于GPU高效的并行计算能力,该方法能够极大限度的提高运算的速度,降低时间的消耗。测试显示:当积分节点数目较多,如$10^6$个节点时,在本文中的测试平台上该方法能够带来218倍的速度提升。

\end{cabstract}

\begin{eabstract}
	The PandaX-III experiment will search for neutrinoless double beta decay of 136 Xe with high pressure gaseous time projection chambers at the China Jin-Ping underground Laboratory. In order to detect such rare event whoes half time longer than $1\times10^{26}$, the experiment need a super low background index with a high signal efficiency. So a proper detector design and a effective Signal-background discrimination are needed for improve the detection sensitivity.

	We simulate background contribution which generated by \utte and \thttt gamma decays in different part of detector based on detailed Monte Carlo simulation. The background mostly comes from Micorbulk Micromegas and stainless bolts which joint detector's copper vessel while others can be ignore. Totally background index of the detector reaches $308\times 10^{-5}$ \ckky.

	Comparing with the background, the track of the neutrinoless double beta decay has a clearly feature,  ends by two Bragg peaks, which can suppress the background level. We study a method based on the convolutional neural networks to discriminate signals against the background. Using the 2-dimensional projections of recorded tracks on two planes, the method successfully suppresses the background level by a factor at 175 with a 47.5\% signal efficiency. A 62\% improvement is achieved in comparison with the baseline in the PandaX-III conceptual design report.

	Except the work in PandaXIII, we also study on the method that using GPU to accelerate the convolution with the resolution as a function of energy. The method can improves the speed as high as 218 times when the integration node is 1 million with our test server, whoes get a significant improvement in the speed without lose RooFit's convenience. 

\end{eabstract}

% vim:ts=4:sw=4
