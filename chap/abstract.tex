% Copyright (c) 2014,2016 Casper Ti. Vector
% Public domain.

\begin{cabstract}
	\pkuthssffaq % 中文测试文字
	PandaXIII 实验在中国锦屏地下实验室建造一个高压气氙时间漂移室(TPC)以寻找\xeots元素的无中微子双 $\beta$ 衰变 (Neutrinoless Double Beta Decay, NLDBD)事件。为了能够探测到这种半衰期大于$1\times10^{26}$年的超稀有事件,探测器需要在保证较高探测效率的同时尽量降低本底水平,即需要通过合理的探测器设计,适当的材料选择以及高效的信号-背景鉴别来确保实验达到较高的灵敏度。

	本文通过蒙特卡洛模拟研究了 PandaXIII 探测器中不同组件含有的 \utte 以及 \thttt 放射性核素级联衰变产生的 $\gamma$ 本底贡献。其中,探测器读出板 Micorbulk Micromegas 以及铆合探测器铜罐的不锈钢螺钉贡献了绝大部分的本底,而探测器的其它组件以及来自周围环境的贡献则可忽略不计。总计探测器本底水平为 $3.08\times 10^{-3}$ count/(keV$\cdot$kg$\cdot$y)。

	相对于本底 $\gamma$ 事件,NLDBD 事件在自身径迹的末端会形成两个布拉格峰,其特征十分鲜明,因此可以通过径迹重建来压低背景信号。本文研究了使用深度卷积神经网络对模拟产生的 NLDBD 信号及 $\gamma$ 背景进行鉴别,该方法能够在保持 47.5\% 信号效率的同时压低背景 175 倍,相较于 PandaXIII 中期设计报告中的目标提升了62\%。

	在 PandaXIII 模拟重建工作之外,本文也探究了在中微子能谱分析研究过程中,考虑探测器分辨率本身随能量连续变化的情况时,使用GPU来加速 Roofit 拟合过程的方法。得益于 GPU 高效的并行计算能力,该方法能够极大程度的提升运算速度,降低时间消耗。测试显示:当积分节点数目较多,如$10^6$个节点时,在本文中的测试平台上(CPU: Intel E5-2603 v3,GPU: Tesla K80)该方法能够带来 218 倍的速度提升。

\end{cabstract}

\begin{eabstract}
    The PandaXIII experiment will search for the neutrinoless double beta decay of $^{136}$Xe with high-pressure gaseous time projection chambers at the China Jin-Ping underground Laboratory. In order to detect such rare event whose half time longer than $1\times10^{26}$, the experiment needs a super-low background index with a high signal efficiency. So a proper detector design and a effective signal-background discrimination are needed to improve the detection sensitivity.

    We simulate background contribution which was generated by \utte~ and \thttt~ gamma decays in different parts of the detector based on detailed Monte Carlo simulation. The background mostly comes from Micorbulk Micromegas and stainless bolts which joins detector's copper vessel while others can be ignored. Totally background index of the detector reaches $308\times 10^{-5}$ \ckky.

    Comparing with the background, the track of the neutrinoless double beta decay has a clearly feature,  ends by two Bragg peaks, which can suppress the background level. We study a method based on the convolutional neural networks to discriminate signals against backgrounds. Using the 2-dimensional projections of recorded tracks on two planes, the method successfully suppresses the background level by a factor of 175 with a 47.5\% signal efficiency. A 62\% improvement is achieved in comparison with the baseline in the PandaX-III conceptual design report.

    Except for the work in PandaX-III, we also study the method that uses GPU to accelerate the convolution with the resolution as a function of energy. The method can improve the speed as high as 218 times when the integration node is 1 million with our test server, whose get a significant improvement in the speed without losing RooFit's convenience. 

\end{eabstract}

% vim:ts=4:sw=4
