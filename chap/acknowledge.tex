% Copyright (c) 2014,2016 Casper Ti. Vector
% Public domain.

\chapter{致谢}
%\pkuthssffaq % 中文测试文字。

时光飞逝,转瞬间我已经在燕园度过了难忘的 7 个春秋,也将完成我的学生生涯,真正的踏入社会中。于此时此地回首过往,在这接近于人生的十分之一的时间里,我从一个茫然无知的高中生逐渐成长,学习了许多也收获了许多,最终留下了无数宝贵的回忆。

在这段快乐学习生活的时光里,我要感谢陪我一起走来的老师,同学,朋友和家人们。首先我要感谢的便是我的导师王思广老师,感谢他在学习和科研上对我的的指导,以及在生活上对我的的关心和照顾。我也要感谢谌勋老师在 PandaXIII 研究中带领我一路走来。这两位老师严谨的治学态度,渊博的学识以及平易近人的人格魅力对我影响深远,是我一生都值得学习的榜样!

其次我要感谢我的同学和舍友们,感谢你们在生活中对我的关照。在这三年愉快的记忆中到处都是你们的身影,谢谢你们的陪伴。我也要感谢父母对我的养育以及毫无保留的支持,感谢你们在我身后的默默付出。你们的健康和幸福是我今生最大的愿望。

最后我也感谢学校对我的培养,祝您百廿生日快乐。
% vim:ts=4:sw=4

\chapter{科研情况统计}

\begin{enumerate}
\item 论文 "Signal-background discrimination with convolutional neural networks in the PandaX-III experiment using MC simulation" 将以第一作者发表于 <SCIENCE CHINA Physics, Mechanics \& Astronomy>, 已接受。
\item 论文 《利用GPU加速信号形状与探测器分辨率随能量变化的卷积》将以第一作者发表于《核技术》, 已接受。
\item 论文 "PandaX-III: Searching for Neutrinoless Double Beta Decay with High Pressure 136Xe Gas Time Projection Chambers" 合作组文章已发表于 <Science China Physics, Mechanics \& Astronomy>, 本人主要参与了其中背景模拟的相关工作。
\item 论文 "Simulation Study of the Performance of New Micro Pattern Gaseous Detectors" 以第四作者发表于 <Radiation Detection Technology and Methods> 。
\end{enumerate}