% Copyright (c) 2014,2016 Casper Ti. Vector
% Public domain.

\specialchap{结论}
\pkuthssffaq % 中文测试文字。

PandaXIII实验是一个使用高压气氙时间漂移室,探寻\xeots无中微子双 Beta 衰变 (NLDBD) 事件的实验。NLDBD是一种极其稀有少见的事件,因而探测它需要极低的探测器本底以及极高的探测器效率。

通过对探测器进行蒙特卡罗模拟,我们得到了不同探测器组件对于本底的贡献。绝大部分的本地来自于MicroMegas读出板以及固定探测器铜罐的螺钉,而其他探测器组件以及环境对本底的贡献要小得多。最终计算得到整个实验的本底水平约为$308\times 10^{-5}c\/(keV\cdot kg\cdot y)$,即对于PandaXIII前期计划的200kg量级探测器而言,每年的本底计数约为78个。

该本底水平还是高于我们的预期,因而我们测试了使用深度卷积神经网络(CNN)进行事件鉴别,以此在不影响探测效率的情况下极大地压低背景。通过训练我们得到了对于蒙特卡洛产生的$\gamma$本底以及NLDBD信号事件,ResNet50网络可以达到175倍的本底压低以及47.5\%的信号效率。其结果超出了我们的预期,比PandaXIII中期设计报告中的目标提升了64\%。结合探测器背景模拟的结果可以给出,PandaXIII实验所设计的1吨量级探测器在经过3年的取数后,可以将\xeots NLDBD事件半衰期下限提升至$1.6\times10^{27}$年。

在PandaXIII模拟以及重建工作外,作者也探究了如何使用GPU加速Roofit拟合过程。因为中微子本征能谱跨度较大,所以在描述探测到的能谱时探测器分辨率本身随能量连续变化应该被考虑。此时传统使用CPU进行拟合分析的方法无法保障时间要求。本文研究了利用GPU来加速数值积分过程,测试显示:当计算节点数目较多,如106个节点时,利用文中的方法GPU(型号:K80)相较于CPU(型号:E2603v3)能够带来218倍的速度提升。


% vim:ts=4:sw=4
