% Copyright (c) 2014,2016 Casper Ti. Vector
% Public domain.

\specialchap{结论}
\pkuthssffaq % 中文测试文字。

PandaXIII 实验是一个使用高压气氙时间漂移室,探寻 \xeots 无中微子双 Beta 衰变 (NLDBD) 事件的实验。NLDBD 极其稀有少见,因而探测它需要极低的探测器本底以及较高的探测器效率。

通过对探测器进行蒙特卡罗模拟,本文得到了不同探测器组件对于本底的贡献。绝大部分的本底信号来自于 MicroMegas 读出板以及固定探测器铜罐的螺钉,而其他探测器组件以及环境对本底的贡献要小得多。最终计算得到整个实验的本底水平约为 $3.08\times 10^{-3}$count\/(keV$\cdot$kg$\cdot$y),即对于 PandaXIII 前期计划的 200kg 量级探测器而言,每年的本底计数约为 78 个。

这一本底水平还是高于实验的预期,因而本文测试了使用深度卷积神经网络进行事件鉴别,以此在不影响探测效率的情况下压低背景。本文使用蒙特卡洛产生的 $\gamma$ 本底以及 NLDBD 信号的训练了 Resnet50 网络,测试显示它可以在压低本地信号 175 倍的同时达到 47.5\% 的信号效率,比 PandaXIII 中期设计报告中的目标提升了 64\%。结合探测器背景模拟的数据后,PandaXIII 实验所设计的 1 吨量级探测器在经过 3 年的取数后,可以将 \xeots NLDBD 事件半衰期下限提升至 $1.6\times10^{27}$ 年。

在 PandaXIII 模拟以及重建工作外,作者也探究了如何使用 GPU 加速 Roofit 拟合过程。因为中微子本征能谱跨度较大,所以在描述探测到的能谱时探测器分辨率本身随能量连续变化的情况应该被考虑。此时传统使用 CPU 进行拟合分析的方法过于缓慢,无法保障时间要求。本文研究了利用 GPU 来加速数值积分过程,测试显示:当计算节点数目较多,如 $10^{6}$ 个节点时,利用文中的方法 GPU(型号:K80)相较于 CPU(型号:E2603v3)能够带来最多 218 倍的速度提升。


% vim:ts=4:sw=4
